%% Generated by Sphinx.
\def\sphinxdocclass{report}
\documentclass[letterpaper,10pt,english]{sphinxmanual}
\ifdefined\pdfpxdimen
   \let\sphinxpxdimen\pdfpxdimen\else\newdimen\sphinxpxdimen
\fi \sphinxpxdimen=.75bp\relax

\PassOptionsToPackage{warn}{textcomp}
\usepackage[utf8]{inputenc}
\ifdefined\DeclareUnicodeCharacter
 \ifdefined\DeclareUnicodeCharacterAsOptional
  \DeclareUnicodeCharacter{"00A0}{\nobreakspace}
  \DeclareUnicodeCharacter{"2500}{\sphinxunichar{2500}}
  \DeclareUnicodeCharacter{"2502}{\sphinxunichar{2502}}
  \DeclareUnicodeCharacter{"2514}{\sphinxunichar{2514}}
  \DeclareUnicodeCharacter{"251C}{\sphinxunichar{251C}}
  \DeclareUnicodeCharacter{"2572}{\textbackslash}
 \else
  \DeclareUnicodeCharacter{00A0}{\nobreakspace}
  \DeclareUnicodeCharacter{2500}{\sphinxunichar{2500}}
  \DeclareUnicodeCharacter{2502}{\sphinxunichar{2502}}
  \DeclareUnicodeCharacter{2514}{\sphinxunichar{2514}}
  \DeclareUnicodeCharacter{251C}{\sphinxunichar{251C}}
  \DeclareUnicodeCharacter{2572}{\textbackslash}
 \fi
\fi
\usepackage{cmap}
\usepackage[T1]{fontenc}
\usepackage{amsmath,amssymb,amstext}
\usepackage{babel}
\usepackage{times}
\usepackage[Bjarne]{fncychap}
\usepackage{sphinx}

\usepackage{geometry}

% Include hyperref last.
\usepackage{hyperref}
% Fix anchor placement for figures with captions.
\usepackage{hypcap}% it must be loaded after hyperref.
% Set up styles of URL: it should be placed after hyperref.
\urlstyle{same}
\addto\captionsenglish{\renewcommand{\contentsname}{Contents:}}

\addto\captionsenglish{\renewcommand{\figurename}{Fig.}}
\addto\captionsenglish{\renewcommand{\tablename}{Table}}
\addto\captionsenglish{\renewcommand{\literalblockname}{Listing}}

\addto\captionsenglish{\renewcommand{\literalblockcontinuedname}{continued from previous page}}
\addto\captionsenglish{\renewcommand{\literalblockcontinuesname}{continues on next page}}

\addto\extrasenglish{\def\pageautorefname{page}}

\setcounter{tocdepth}{1}



\title{Documentation new TSI software Documentation}
\date{May 25, 2018}
\release{0.0.1}
\author{Job Mos}
\newcommand{\sphinxlogo}{\vbox{}}
\renewcommand{\releasename}{Release}
\makeindex

\begin{document}

\maketitle
\sphinxtableofcontents
\phantomsection\label{\detokenize{index::doc}}



\chapter{Introduction}
\label{\detokenize{intro::doc}}\label{\detokenize{intro:welcome-to-documentation-new-tsi-software-s-documentation}}\label{\detokenize{intro:introduction}}
This is the introduction.


\chapter{Code example}
\label{\detokenize{code_example::doc}}\label{\detokenize{code_example:code-example}}
Here is a Python function.

\fvset{hllines={, ,}}%
\begin{sphinxVerbatim}[commandchars=\\\{\}]
\PYG{k}{def} \PYG{n+nf}{greet}\PYG{p}{(}\PYG{n}{name}\PYG{p}{)}\PYG{p}{:}
    \PYG{k}{print}\PYG{p}{(}\PYG{l+s+s2}{\PYGZdq{}}\PYG{l+s+s2}{Hello \PYGZob{}\PYGZcb{}}\PYG{l+s+s2}{\PYGZdq{}}\PYG{o}{.}\PYG{n}{format}\PYG{p}{(}\PYG{n}{name}\PYG{p}{)}\PYG{p}{)}
\end{sphinxVerbatim}

Here is a C function.

\fvset{hllines={, ,}}%
\begin{sphinxVerbatim}[commandchars=\\\{\}]
\PYG{k+kt}{int} \PYG{n+nf}{add}\PYG{p}{(}\PYG{k+kt}{int} \PYG{n}{a}\PYG{p}{,} \PYG{k+kt}{int} \PYG{n}{b}\PYG{p}{)} \PYG{p}{\PYGZob{}}
    \PYG{k}{return} \PYG{n}{a} \PYG{o}{+} \PYG{n}{b}\PYG{p}{;}
\PYG{p}{\PYGZcb{}}
\end{sphinxVerbatim}

\fvset{hllines={, ,}}%
\begin{sphinxVerbatim}[commandchars=\\\{\}]
\PYG{l+s+sd}{\PYGZdq{}\PYGZdq{}\PYGZdq{}This is an example script.\PYGZdq{}\PYGZdq{}\PYGZdq{}}
\PYG{k+kn}{import} \PYG{n+nn}{sys}

\PYG{k}{def} \PYG{n+nf}{greet}\PYG{p}{(}\PYG{n}{name}\PYG{p}{)}\PYG{p}{:}
    \PYG{l+s+sd}{\PYGZdq{}\PYGZdq{}\PYGZdq{}Return greeting.\PYGZdq{}\PYGZdq{}\PYGZdq{}}
    \PYG{k}{return} \PYG{l+s+s2}{\PYGZdq{}}\PYG{l+s+s2}{Hello \PYGZob{}\PYGZcb{}!}\PYG{l+s+s2}{\PYGZdq{}}\PYG{o}{.}\PYG{n}{format}\PYG{p}{(}\PYG{n}{name}\PYG{p}{)}

\PYG{k}{if} \PYG{n+nv+vm}{\PYGZus{}\PYGZus{}name\PYGZus{}\PYGZus{}} \PYG{o}{==} \PYG{l+s+s2}{\PYGZdq{}}\PYG{l+s+s2}{\PYGZus{}\PYGZus{}main\PYGZus{}\PYGZus{}}\PYG{l+s+s2}{\PYGZdq{}}\PYG{p}{:}
    \PYG{n}{name} \PYG{o}{=} \PYG{n}{sys}\PYG{o}{.}\PYG{n}{argv}\PYG{p}{[}\PYG{l+m+mi}{1}\PYG{p}{]}
    \PYG{k}{print}\PYG{p}{(}\PYG{n}{greet}\PYG{p}{(}\PYG{n}{name}\PYG{p}{)}\PYG{p}{)}
\end{sphinxVerbatim}


\chapter{cloudDetection}
\label{\detokenize{modules::doc}}\label{\detokenize{modules:clouddetection}}

\section{color\_bands module}
\label{\detokenize{color_bands:module-color_bands}}\label{\detokenize{color_bands:color-bands-module}}\label{\detokenize{color_bands::doc}}\index{color\_bands (module)}\index{extract() (in module color\_bands)}

\begin{fulllineitems}
\phantomsection\label{\detokenize{color_bands:color_bands.extract}}\pysiglinewithargsret{\sphinxcode{\sphinxupquote{color\_bands.}}\sphinxbfcode{\sphinxupquote{extract}}}{\emph{scaler}, \emph{masked\_img}}{}
Extract the red, green and blue bands from the masked image
\begin{quote}\begin{description}
\item[{Parameters}] \leavevmode\begin{itemize}
\item {} 
\sphinxstyleliteralstrong{\sphinxupquote{scaler}} (\sphinxstyleliteralemphasis{\sphinxupquote{int}}) \textendash{} Maximum number of color levels, used in GLCM matrix

\item {} 
\sphinxstyleliteralstrong{\sphinxupquote{masked\_img}} (\sphinxstyleliteralemphasis{\sphinxupquote{int}}) \textendash{} The masked image

\end{itemize}

\item[{Returns}] \leavevmode
Red, green and blue bands

\item[{Return type}] \leavevmode
tuple

\end{description}\end{quote}

\end{fulllineitems}



\section{createmask module}
\label{\detokenize{createmask:module-createmask}}\label{\detokenize{createmask::doc}}\label{\detokenize{createmask:createmask-module}}\index{createmask (module)}\index{calculate\_band\_position() (in module createmask)}

\begin{fulllineitems}
\phantomsection\label{\detokenize{createmask:createmask.calculate_band_position}}\pysiglinewithargsret{\sphinxcode{\sphinxupquote{createmask.}}\sphinxbfcode{\sphinxupquote{calculate\_band\_position}}}{\emph{theta}}{}
Calculate the inner and outer position of the shadow band, required for drawing the shadow band mask line.

The formula of a circle is used: \(x = r \cos{\theta} \wedge y = r \sin{\theta}\).
\begin{quote}\begin{description}
\item[{Parameters}] \leavevmode
\sphinxstyleliteralstrong{\sphinxupquote{theta}} (\sphinxstyleliteralemphasis{\sphinxupquote{float}}) \textendash{} Azimuth of the sun with respect to the East. Normally, azimuth is measured from the North. However,
to simplify calculations, the east was used.

\item[{Returns}] \leavevmode
X and y locations of the inner and outer points of the shadow band

\item[{Return type}] \leavevmode
tuple

\end{description}\end{quote}

\end{fulllineitems}

\index{create() (in module createmask)}

\begin{fulllineitems}
\phantomsection\label{\detokenize{createmask:createmask.create}}\pysiglinewithargsret{\sphinxcode{\sphinxupquote{createmask.}}\sphinxbfcode{\sphinxupquote{create}}}{\emph{img}, \emph{azimuth}}{}
Create the mask using the original image and the azimuth.

The mask consists of three parts:
\begin{itemize}
\item {} 
The circle bordering the hemispherical mirror.

\item {} 
The shadow band.

\item {} 
The camera and camera arm.

\end{itemize}
\begin{quote}\begin{description}
\item[{Parameters}] \leavevmode\begin{itemize}
\item {} 
\sphinxstyleliteralstrong{\sphinxupquote{img}} (\sphinxstyleliteralemphasis{\sphinxupquote{int}}) \textendash{} Image in NumPy format

\item {} 
\sphinxstyleliteralstrong{\sphinxupquote{azimuth}} (\sphinxstyleliteralemphasis{\sphinxupquote{float}}) \textendash{} Azimuth of the sun, taken from the properties file

\end{itemize}

\item[{Returns}] \leavevmode
The masked image of shape (x\_resolution,y\_resolution,3) for an RGB image

\item[{Return type}] \leavevmode
int

\end{description}\end{quote}

\end{fulllineitems}



\section{createregions module}
\label{\detokenize{createregions:module-createregions}}\label{\detokenize{createregions::doc}}\label{\detokenize{createregions:createregions-module}}\index{createregions (module)}\index{create() (in module createregions)}

\begin{fulllineitems}
\phantomsection\label{\detokenize{createregions:createregions.create}}\pysiglinewithargsret{\sphinxcode{\sphinxupquote{createregions.}}\sphinxbfcode{\sphinxupquote{create}}}{\emph{img}, \emph{azimuth}, \emph{altitude}}{}
Create the empty arrays and call drawing and masking functions
\begin{quote}\begin{description}
\item[{Parameters}] \leavevmode\begin{itemize}
\item {} 
\sphinxstyleliteralstrong{\sphinxupquote{img}} (\sphinxstyleliteralemphasis{\sphinxupquote{int}}) \textendash{} image in NumPy format

\item {} 
\sphinxstyleliteralstrong{\sphinxupquote{azimuth}} (\sphinxstyleliteralemphasis{\sphinxupquote{float}}) \textendash{} azimuth of the sun, taken from the properties file

\item {} 
\sphinxstyleliteralstrong{\sphinxupquote{altitude}} (\sphinxstyleliteralemphasis{\sphinxupquote{float}}) \textendash{} altitude of the sun, taken from the properties file

\end{itemize}

\item[{Returns}] \leavevmode
regions, outlines, labels, stencil, image\_with\_outlines

\item[{Return type}] \leavevmode
tuple

\end{description}\end{quote}

\end{fulllineitems}

\index{create\_stencil() (in module createregions)}

\begin{fulllineitems}
\phantomsection\label{\detokenize{createregions:createregions.create_stencil}}\pysiglinewithargsret{\sphinxcode{\sphinxupquote{createregions.}}\sphinxbfcode{\sphinxupquote{create\_stencil}}}{\emph{stencil}, \emph{stencil\_labels}}{}
Create the stencil which is used to mask the outside of the large circle
\begin{quote}\begin{description}
\item[{Parameters}] \leavevmode\begin{itemize}
\item {} 
\sphinxstyleliteralstrong{\sphinxupquote{stencil}} (\sphinxstyleliteralemphasis{\sphinxupquote{int}}) \textendash{} empty stencil array in RGB format

\item {} 
\sphinxstyleliteralstrong{\sphinxupquote{stencil\_labels}} (\sphinxstyleliteralemphasis{\sphinxupquote{int}}) \textendash{} empty stencil array in scalar format

\end{itemize}

\item[{Returns}] \leavevmode
stencil in both RGB and scalar format

\item[{Return type}] \leavevmode
tuple

\end{description}\end{quote}

\end{fulllineitems}

\index{draw\_arm() (in module createregions)}

\begin{fulllineitems}
\phantomsection\label{\detokenize{createregions:createregions.draw_arm}}\pysiglinewithargsret{\sphinxcode{\sphinxupquote{createregions.}}\sphinxbfcode{\sphinxupquote{draw\_arm}}}{\emph{regions}, \emph{labels}, \emph{image\_with\_outlines}}{}
Draw the camera arm mask
\begin{quote}\begin{description}
\item[{Parameters}] \leavevmode\begin{itemize}
\item {} 
\sphinxstyleliteralstrong{\sphinxupquote{regions}} (\sphinxstyleliteralemphasis{\sphinxupquote{int}}) \textendash{} RGB representation of the segmented image

\item {} 
\sphinxstyleliteralstrong{\sphinxupquote{labels}} (\sphinxstyleliteralemphasis{\sphinxupquote{int}}) \textendash{} Scalar (1,2,3,4) representation of the segmented image

\item {} 
\sphinxstyleliteralstrong{\sphinxupquote{image\_with\_outlines}} (\sphinxstyleliteralemphasis{\sphinxupquote{int}}) \textendash{} image with outlines as overlay

\end{itemize}

\item[{Returns}] \leavevmode
regions, labels, image\_with\_outlines

\item[{Return type}] \leavevmode
tuple

\end{description}\end{quote}

\end{fulllineitems}

\index{draw\_band() (in module createregions)}

\begin{fulllineitems}
\phantomsection\label{\detokenize{createregions:createregions.draw_band}}\pysiglinewithargsret{\sphinxcode{\sphinxupquote{createregions.}}\sphinxbfcode{\sphinxupquote{draw\_band}}}{\emph{regions}, \emph{labels}, \emph{image\_with\_outlines}, \emph{theta}}{}
Draw the shadow band mask
\begin{quote}\begin{description}
\item[{Parameters}] \leavevmode\begin{itemize}
\item {} 
\sphinxstyleliteralstrong{\sphinxupquote{regions}} (\sphinxstyleliteralemphasis{\sphinxupquote{int}}) \textendash{} RGB representation of the segmented image

\item {} 
\sphinxstyleliteralstrong{\sphinxupquote{labels}} (\sphinxstyleliteralemphasis{\sphinxupquote{int}}) \textendash{} Scalar (1,2,3,4) representation of the segmented image

\item {} 
\sphinxstyleliteralstrong{\sphinxupquote{image\_with\_outlines}} (\sphinxstyleliteralemphasis{\sphinxupquote{int}}) \textendash{} image with outlines as overlay

\item {} 
\sphinxstyleliteralstrong{\sphinxupquote{theta}} (\sphinxstyleliteralemphasis{\sphinxupquote{float}}) \textendash{} azimuth measured from the East

\end{itemize}

\item[{Returns}] \leavevmode
regions, labels, image\_with\_outlines

\end{description}\end{quote}

\end{fulllineitems}

\index{draw\_horizon\_area() (in module createregions)}

\begin{fulllineitems}
\phantomsection\label{\detokenize{createregions:createregions.draw_horizon_area}}\pysiglinewithargsret{\sphinxcode{\sphinxupquote{createregions.}}\sphinxbfcode{\sphinxupquote{draw\_horizon\_area}}}{\emph{azimuth}, \emph{regions}, \emph{labels}, \emph{outlines}}{}
Draw polygon which (when combined with other segments) makes up the horizon area
\begin{quote}\begin{description}
\item[{Parameters}] \leavevmode\begin{itemize}
\item {} 
\sphinxstyleliteralstrong{\sphinxupquote{azimuth}} (\sphinxstyleliteralemphasis{\sphinxupquote{float}}) \textendash{} solar azimuth in degrees from North

\item {} 
\sphinxstyleliteralstrong{\sphinxupquote{regions}} (\sphinxstyleliteralemphasis{\sphinxupquote{int}}) \textendash{} RGB representation of the segmented image

\item {} 
\sphinxstyleliteralstrong{\sphinxupquote{labels}} (\sphinxstyleliteralemphasis{\sphinxupquote{int}}) \textendash{} Scalar (1,2,3,4) representation of the segmented image

\item {} 
\sphinxstyleliteralstrong{\sphinxupquote{outlines}} (\sphinxstyleliteralemphasis{\sphinxupquote{int}}) \textendash{} RGB array of the segment outlines

\end{itemize}

\item[{Returns}] \leavevmode
Regions, labels, outlines and angle \(\theta\) describing the azimuth measured from the East

\item[{Return type}] \leavevmode
tuple

\end{description}\end{quote}

\end{fulllineitems}

\index{inner\_circle() (in module createregions)}

\begin{fulllineitems}
\phantomsection\label{\detokenize{createregions:createregions.inner_circle}}\pysiglinewithargsret{\sphinxcode{\sphinxupquote{createregions.}}\sphinxbfcode{\sphinxupquote{inner\_circle}}}{\emph{regions}, \emph{labels}, \emph{outlines}}{}
Draw the inner circle of the segmented image

The radius is smaller than the radius used in {\hyperref[\detokenize{createregions:createregions.large_circle}]{\sphinxcrossref{\sphinxcode{\sphinxupquote{createregions.large\_circle()}}}}}
\begin{quote}\begin{description}
\item[{Parameters}] \leavevmode\begin{itemize}
\item {} 
\sphinxstyleliteralstrong{\sphinxupquote{regions}} (\sphinxstyleliteralemphasis{\sphinxupquote{int}}) \textendash{} RGB representation of the segmented image

\item {} 
\sphinxstyleliteralstrong{\sphinxupquote{labels}} (\sphinxstyleliteralemphasis{\sphinxupquote{int}}) \textendash{} Scalar (1,2,3,4) representation of the segmented image

\item {} 
\sphinxstyleliteralstrong{\sphinxupquote{outlines}} (\sphinxstyleliteralemphasis{\sphinxupquote{int}}) \textendash{} RGB array of the segment outlines

\end{itemize}

\item[{Returns}] \leavevmode
regions, labels, outlines

\item[{Return type}] \leavevmode
tuple

\end{description}\end{quote}

\end{fulllineitems}

\index{large\_circle() (in module createregions)}

\begin{fulllineitems}
\phantomsection\label{\detokenize{createregions:createregions.large_circle}}\pysiglinewithargsret{\sphinxcode{\sphinxupquote{createregions.}}\sphinxbfcode{\sphinxupquote{large\_circle}}}{\emph{regions}, \emph{labels}, \emph{outlines}}{}
Draw a circle centered in the middle of the image.

This circle has a radius slightly smaller than that of the mirror.
\begin{quote}\begin{description}
\item[{Parameters}] \leavevmode\begin{itemize}
\item {} 
\sphinxstyleliteralstrong{\sphinxupquote{regions}} (\sphinxstyleliteralemphasis{\sphinxupquote{int}}) \textendash{} RGB representation of the segmented image

\item {} 
\sphinxstyleliteralstrong{\sphinxupquote{labels}} (\sphinxstyleliteralemphasis{\sphinxupquote{int}}) \textendash{} Scalar (1,2,3,4) representation of the segmented image

\item {} 
\sphinxstyleliteralstrong{\sphinxupquote{outlines}} (\sphinxstyleliteralemphasis{\sphinxupquote{int}}) \textendash{} RGB array of the segment outlines

\end{itemize}

\item[{Returns}] \leavevmode
regions, labels, outlines

\item[{Return type}] \leavevmode
tuple

\end{description}\end{quote}

\end{fulllineitems}

\index{outer\_circle() (in module createregions)}

\begin{fulllineitems}
\phantomsection\label{\detokenize{createregions:createregions.outer_circle}}\pysiglinewithargsret{\sphinxcode{\sphinxupquote{createregions.}}\sphinxbfcode{\sphinxupquote{outer\_circle}}}{\emph{regions}, \emph{labels}, \emph{outlines}, \emph{stencil}, \emph{stencil\_labels}}{}
Mask the outside of the large circle
\begin{quote}\begin{description}
\item[{Parameters}] \leavevmode\begin{itemize}
\item {} 
\sphinxstyleliteralstrong{\sphinxupquote{regions}} (\sphinxstyleliteralemphasis{\sphinxupquote{int}}) \textendash{} RGB representation of the segmented image

\item {} 
\sphinxstyleliteralstrong{\sphinxupquote{labels}} (\sphinxstyleliteralemphasis{\sphinxupquote{int}}) \textendash{} Scalar (1,2,3,4) representation of the segmented image

\item {} 
\sphinxstyleliteralstrong{\sphinxupquote{outlines}} (\sphinxstyleliteralemphasis{\sphinxupquote{int}}) \textendash{} RGB array of the segment outlines

\item {} 
\sphinxstyleliteralstrong{\sphinxupquote{stencil}} (\sphinxstyleliteralemphasis{\sphinxupquote{int}}) \textendash{} stencil array in RGB format

\item {} 
\sphinxstyleliteralstrong{\sphinxupquote{stencil\_labels}} (\sphinxstyleliteralemphasis{\sphinxupquote{int}}) \textendash{} stencil array in scalar format

\end{itemize}

\item[{Returns}] \leavevmode
regions, labels, outlines, stencil (RGB), stencil (scalar)

\item[{Return type}] \leavevmode
tuple

\end{description}\end{quote}

\end{fulllineitems}

\index{overlay\_outlines\_on\_image() (in module createregions)}

\begin{fulllineitems}
\phantomsection\label{\detokenize{createregions:createregions.overlay_outlines_on_image}}\pysiglinewithargsret{\sphinxcode{\sphinxupquote{createregions.}}\sphinxbfcode{\sphinxupquote{overlay\_outlines\_on\_image}}}{\emph{img}, \emph{outlines}, \emph{stencil}}{}
Overlay outlines on image by converting to BW and performing several other operations
\begin{quote}\begin{description}
\item[{Parameters}] \leavevmode\begin{itemize}
\item {} 
\sphinxstyleliteralstrong{\sphinxupquote{img}} (\sphinxstyleliteralemphasis{\sphinxupquote{int}}) \textendash{} image in NumPy format

\item {} 
\sphinxstyleliteralstrong{\sphinxupquote{outlines}} (\sphinxstyleliteralemphasis{\sphinxupquote{int}}) \textendash{} RGB array of the segment outlines

\item {} 
\sphinxstyleliteralstrong{\sphinxupquote{stencil}} (\sphinxstyleliteralemphasis{\sphinxupquote{int}}) \textendash{} stencil array in RGB format

\end{itemize}

\item[{Returns}] \leavevmode
image with outlines as overlay

\item[{Return type}] \leavevmode
int

\end{description}\end{quote}

\end{fulllineitems}

\index{sun\_circle() (in module createregions)}

\begin{fulllineitems}
\phantomsection\label{\detokenize{createregions:createregions.sun_circle}}\pysiglinewithargsret{\sphinxcode{\sphinxupquote{createregions.}}\sphinxbfcode{\sphinxupquote{sun\_circle}}}{\emph{altitude}, \emph{regions}, \emph{labels}, \emph{outlines}, \emph{theta}}{}
Draw the sun cirlce segment

The position of the sun in the image plane is calculated using an approximation of the mirror. The function that is
used to estimate the mirror geometry is \(y = -0.23x+1.25\).

The radial distance from the center of the image to the center of the sun can subsequently be calculated using
the quadratic equation (abc formula).

Using the description of a circle ({\hyperref[\detokenize{createmask:createmask.calculate_band_position}]{\sphinxcrossref{\sphinxcode{\sphinxupquote{createmask.calculate\_band\_position()}}}}}), the solar position is calculated.
\begin{quote}\begin{description}
\item[{Parameters}] \leavevmode\begin{itemize}
\item {} 
\sphinxstyleliteralstrong{\sphinxupquote{altitude}} (\sphinxstyleliteralemphasis{\sphinxupquote{float}}) \textendash{} altitude of the sun, taken from the properties file

\item {} 
\sphinxstyleliteralstrong{\sphinxupquote{regions}} (\sphinxstyleliteralemphasis{\sphinxupquote{int}}) \textendash{} RGB representation of the segmented image

\item {} 
\sphinxstyleliteralstrong{\sphinxupquote{labels}} (\sphinxstyleliteralemphasis{\sphinxupquote{int}}) \textendash{} Scalar (1,2,3,4) representation of the segmented image

\item {} 
\sphinxstyleliteralstrong{\sphinxupquote{outlines}} (\sphinxstyleliteralemphasis{\sphinxupquote{int}}) \textendash{} RGB array of the segment outlines

\item {} 
\sphinxstyleliteralstrong{\sphinxupquote{theta}} (\sphinxstyleliteralemphasis{\sphinxupquote{float}}) \textendash{} azimuth measured from the East

\end{itemize}

\item[{Returns}] \leavevmode
regions, labels, outlines

\item[{Return type}] \leavevmode
tuple

\end{description}\end{quote}

\end{fulllineitems}



\section{labelled\_image module}
\label{\detokenize{labelled_image:module-labelled_image}}\label{\detokenize{labelled_image::doc}}\label{\detokenize{labelled_image:labelled-image-module}}\index{labelled\_image (module)}\index{calculate\_pixels() (in module labelled\_image)}

\begin{fulllineitems}
\phantomsection\label{\detokenize{labelled_image:labelled_image.calculate_pixels}}\pysiglinewithargsret{\sphinxcode{\sphinxupquote{labelled\_image.}}\sphinxbfcode{\sphinxupquote{calculate\_pixels}}}{\emph{labels}, \emph{red\_blue\_ratio}, \emph{threshold}}{}
Get amount of pixels in the four different areas to be used in postprocessing corrections
\begin{quote}\begin{description}
\item[{Parameters}] \leavevmode\begin{itemize}
\item {} 
\sphinxstyleliteralstrong{\sphinxupquote{labels}} (\sphinxstyleliteralemphasis{\sphinxupquote{int}}) \textendash{} Scalar representation of the segmented image

\item {} 
\sphinxstyleliteralstrong{\sphinxupquote{red\_blue\_ratio}} (\sphinxstyleliteralemphasis{\sphinxupquote{float}}) \textendash{} ratio of red/blue bands

\item {} 
\sphinxstyleliteralstrong{\sphinxupquote{threshold}} (\sphinxstyleliteralemphasis{\sphinxupquote{float}}) \textendash{} fixed threshold of sunny/cloudy

\end{itemize}

\item[{Returns}] \leavevmode
amount of sunny and cloudy pixels in each of the four regions

\item[{Return type}] \leavevmode
tuple

\end{description}\end{quote}

\end{fulllineitems}



\section{main module}
\label{\detokenize{main:module-main}}\label{\detokenize{main:main-module}}\label{\detokenize{main::doc}}\index{main (module)}\index{main() (in module main)}

\begin{fulllineitems}
\phantomsection\label{\detokenize{main:main.main}}\pysiglinewithargsret{\sphinxcode{\sphinxupquote{main.}}\sphinxbfcode{\sphinxupquote{main}}}{}{}
Call processing functions and write output to file

\end{fulllineitems}



\section{myimports module}
\label{\detokenize{myimports:module-myimports}}\label{\detokenize{myimports::doc}}\label{\detokenize{myimports:myimports-module}}\index{myimports (module)}

\section{overlay module}
\label{\detokenize{overlay::doc}}\label{\detokenize{overlay:overlay-module}}\label{\detokenize{overlay:module-overlay}}\index{overlay (module)}\index{fixed() (in module overlay)}

\begin{fulllineitems}
\phantomsection\label{\detokenize{overlay:overlay.fixed}}\pysiglinewithargsret{\sphinxcode{\sphinxupquote{overlay.}}\sphinxbfcode{\sphinxupquote{fixed}}}{\emph{img}, \emph{outlines}, \emph{stencil}, \emph{fixed\_sunny\_threshold}, \emph{fixed\_thin\_threshold}}{}
Preprocess image to be compatible with {\hyperref[\detokenize{overlay:overlay.outlines_over_image}]{\sphinxcrossref{\sphinxcode{\sphinxupquote{overlay.outlines\_over\_image()}}}}} using fixed thresholding
\begin{quote}\begin{description}
\item[{Parameters}] \leavevmode\begin{itemize}
\item {} 
\sphinxstyleliteralstrong{\sphinxupquote{img}} (\sphinxstyleliteralemphasis{\sphinxupquote{int}}) \textendash{} image in NumPy format

\item {} 
\sphinxstyleliteralstrong{\sphinxupquote{outlines}} (\sphinxstyleliteralemphasis{\sphinxupquote{int}}) \textendash{} RGB array of the segment outlines

\item {} 
\sphinxstyleliteralstrong{\sphinxupquote{stencil}} (\sphinxstyleliteralemphasis{\sphinxupquote{int}}) \textendash{} stencil array in RGB format

\item {} 
\sphinxstyleliteralstrong{\sphinxupquote{fixed\_sunny\_threshold}} (\sphinxstyleliteralemphasis{\sphinxupquote{float}}) \textendash{} threshold for sun/cloud

\item {} 
\sphinxstyleliteralstrong{\sphinxupquote{fixed\_thin\_threshold}} (\sphinxstyleliteralemphasis{\sphinxupquote{float}}) \textendash{} threshold for thin/opaque cloud

\end{itemize}

\item[{Returns}] \leavevmode
image with outlines

\item[{Return type}] \leavevmode
int

\end{description}\end{quote}

\end{fulllineitems}

\index{hybrid() (in module overlay)}

\begin{fulllineitems}
\phantomsection\label{\detokenize{overlay:overlay.hybrid}}\pysiglinewithargsret{\sphinxcode{\sphinxupquote{overlay.}}\sphinxbfcode{\sphinxupquote{hybrid}}}{\emph{img}, \emph{outlines}, \emph{stencil}, \emph{threshold}}{}
Preprocess image to be compatible with {\hyperref[\detokenize{overlay:overlay.outlines_over_image}]{\sphinxcrossref{\sphinxcode{\sphinxupquote{overlay.outlines\_over\_image()}}}}} using the hybrid threshold
\begin{quote}\begin{description}
\item[{Parameters}] \leavevmode\begin{itemize}
\item {} 
\sphinxstyleliteralstrong{\sphinxupquote{img}} (\sphinxstyleliteralemphasis{\sphinxupquote{int}}) \textendash{} image in NumPy format

\item {} 
\sphinxstyleliteralstrong{\sphinxupquote{outlines}} (\sphinxstyleliteralemphasis{\sphinxupquote{int}}) \textendash{} RGB array of the segment outlines

\item {} 
\sphinxstyleliteralstrong{\sphinxupquote{stencil}} (\sphinxstyleliteralemphasis{\sphinxupquote{int}}) \textendash{} stencil array in RGB format

\item {} 
\sphinxstyleliteralstrong{\sphinxupquote{threshold}} (\sphinxstyleliteralemphasis{\sphinxupquote{float}}) \textendash{} threshold for sun/cloud determined by HYbrid Thresholding Algorithm (HYTA)

\end{itemize}

\item[{Returns}] \leavevmode
image with outlines

\item[{Return type}] \leavevmode
int

\end{description}\end{quote}

\end{fulllineitems}

\index{outlines\_over\_image() (in module overlay)}

\begin{fulllineitems}
\phantomsection\label{\detokenize{overlay:overlay.outlines_over_image}}\pysiglinewithargsret{\sphinxcode{\sphinxupquote{overlay.}}\sphinxbfcode{\sphinxupquote{outlines\_over\_image}}}{\emph{img}, \emph{outlines}, \emph{stencil}}{}
Overlay outlines on image by converting to BW and performing several other operations
\begin{quote}\begin{description}
\item[{Parameters}] \leavevmode\begin{itemize}
\item {} 
\sphinxstyleliteralstrong{\sphinxupquote{img}} (\sphinxstyleliteralemphasis{\sphinxupquote{int}}) \textendash{} image in NumPy format

\item {} 
\sphinxstyleliteralstrong{\sphinxupquote{outlines}} (\sphinxstyleliteralemphasis{\sphinxupquote{int}}) \textendash{} RGB array of the segment outlines

\item {} 
\sphinxstyleliteralstrong{\sphinxupquote{stencil}} (\sphinxstyleliteralemphasis{\sphinxupquote{int}}) \textendash{} stencil array in RGB format

\end{itemize}

\item[{Returns}] \leavevmode
image with outlines as overlay

\item[{Return type}] \leavevmode
int

\end{description}\end{quote}

\end{fulllineitems}



\section{postprocessor module}
\label{\detokenize{postprocessor:module-postprocessor}}\label{\detokenize{postprocessor::doc}}\label{\detokenize{postprocessor:postprocessor-module}}\index{postprocessor (module)}\index{aerosol\_correction() (in module postprocessor)}

\begin{fulllineitems}
\phantomsection\label{\detokenize{postprocessor:postprocessor.aerosol_correction}}\pysiglinewithargsret{\sphinxcode{\sphinxupquote{postprocessor.}}\sphinxbfcode{\sphinxupquote{aerosol\_correction}}}{}{}
Perform the horizon area/sun circle correction.

The data from the main processing loop is used which is then subjected to a few steps. The approach by Long 2010
is used. Several statistical features of the segmetns are tested against a set of thresholds defined in
{\hyperref[\detokenize{settings:module-settings}]{\sphinxcrossref{\sphinxcode{\sphinxupquote{settings()}}}}}. Subsequently, the corrected sky cover percentages are written to a file.

\end{fulllineitems}



\section{processor module}
\label{\detokenize{processor:module-processor}}\label{\detokenize{processor::doc}}\label{\detokenize{processor:processor-module}}\index{processor (module)}\index{processor() (in module processor)}

\begin{fulllineitems}
\phantomsection\label{\detokenize{processor:processor.processor}}\pysiglinewithargsret{\sphinxcode{\sphinxupquote{processor.}}\sphinxbfcode{\sphinxupquote{processor}}}{\emph{img}, \emph{img\_tsi}, \emph{azimuth}, \emph{altitude}, \emph{filename}}{}
Call underlying processing routines and return the information to {\hyperref[\detokenize{main:module-main}]{\sphinxcrossref{\sphinxcode{\sphinxupquote{main()}}}}}.
\begin{quote}\begin{description}
\item[{Parameters}] \leavevmode\begin{itemize}
\item {} 
\sphinxstyleliteralstrong{\sphinxupquote{img}} (\sphinxstyleliteralemphasis{\sphinxupquote{int}}) \textendash{} Original image

\item {} 
\sphinxstyleliteralstrong{\sphinxupquote{img\_tsi}} (\sphinxstyleliteralemphasis{\sphinxupquote{int}}) \textendash{} Processed image from the tsi software

\item {} 
\sphinxstyleliteralstrong{\sphinxupquote{azimuth}} (\sphinxstyleliteralemphasis{\sphinxupquote{float}}) \textendash{} azimuth of the sun, taken from the properties file

\item {} 
\sphinxstyleliteralstrong{\sphinxupquote{altitude}} (\sphinxstyleliteralemphasis{\sphinxupquote{float}}) \textendash{} altitude of the sun, taken from the properties file

\item {} 
\sphinxstyleliteralstrong{\sphinxupquote{filename}} (\sphinxstyleliteralemphasis{\sphinxupquote{str}}) \textendash{} Name of the file currently in use

\end{itemize}

\item[{Returns}] \leavevmode
sky cover percentages, masked image, and pixel counts

\item[{Return type}] \leavevmode
tuple

\end{description}\end{quote}

\end{fulllineitems}



\section{ratio module}
\label{\detokenize{ratio:ratio-module}}\label{\detokenize{ratio::doc}}\label{\detokenize{ratio:module-ratio}}\index{ratio (module)}\index{red\_blue() (in module ratio)}

\begin{fulllineitems}
\phantomsection\label{\detokenize{ratio:ratio.red_blue}}\pysiglinewithargsret{\sphinxcode{\sphinxupquote{ratio.}}\sphinxbfcode{\sphinxupquote{red\_blue}}}{\emph{maskedImg}}{}
Calculate the red/blue ratio per image pixel
\begin{quote}\begin{description}
\item[{Parameters}] \leavevmode
\sphinxstyleliteralstrong{\sphinxupquote{maskedImg}} (\sphinxstyleliteralemphasis{\sphinxupquote{int}}) \textendash{} masked image

\item[{Returns}] \leavevmode
red/blue ratio per image pixel

\item[{Return type}] \leavevmode
float

\end{description}\end{quote}

\end{fulllineitems}



\section{read\_properties\_file module}
\label{\detokenize{read_properties_file:read-properties-file-module}}\label{\detokenize{read_properties_file:module-read_properties_file}}\label{\detokenize{read_properties_file::doc}}\index{read\_properties\_file (module)}\index{get\_altitude() (in module read\_properties\_file)}

\begin{fulllineitems}
\phantomsection\label{\detokenize{read_properties_file:read_properties_file.get_altitude}}\pysiglinewithargsret{\sphinxcode{\sphinxupquote{read\_properties\_file.}}\sphinxbfcode{\sphinxupquote{get\_altitude}}}{\emph{lines}}{}
Get the solar altitude from the TSI properties file
\begin{quote}\begin{description}
\item[{Parameters}] \leavevmode
\sphinxstyleliteralstrong{\sphinxupquote{lines}} \textendash{} line by line reading of the properties file

\item[{Returns}] \leavevmode
altitude of the sun

\end{description}\end{quote}

\end{fulllineitems}

\index{get\_azimuth() (in module read\_properties\_file)}

\begin{fulllineitems}
\phantomsection\label{\detokenize{read_properties_file:read_properties_file.get_azimuth}}\pysiglinewithargsret{\sphinxcode{\sphinxupquote{read\_properties\_file.}}\sphinxbfcode{\sphinxupquote{get\_azimuth}}}{\emph{lines}}{}
Get the solar azimuth from the TSI properties file
\begin{quote}\begin{description}
\item[{Parameters}] \leavevmode
\sphinxstyleliteralstrong{\sphinxupquote{lines}} \textendash{} line by line reading of the properties file

\item[{Returns}] \leavevmode
azimuth of the sun

\end{description}\end{quote}

\end{fulllineitems}

\index{get\_fractional\_sky\_cover\_tsi() (in module read\_properties\_file)}

\begin{fulllineitems}
\phantomsection\label{\detokenize{read_properties_file:read_properties_file.get_fractional_sky_cover_tsi}}\pysiglinewithargsret{\sphinxcode{\sphinxupquote{read\_properties\_file.}}\sphinxbfcode{\sphinxupquote{get\_fractional\_sky\_cover\_tsi}}}{\emph{lines}}{}
Get the fractional sky cover from the TSI properties file
\begin{quote}\begin{description}
\item[{Parameters}] \leavevmode
\sphinxstyleliteralstrong{\sphinxupquote{lines}} \textendash{} line by line reading of the properties file

\item[{Returns}] \leavevmode
fractional sky cover

\end{description}\end{quote}

\end{fulllineitems}



\section{resolution module}
\label{\detokenize{resolution:resolution-module}}\label{\detokenize{resolution::doc}}\label{\detokenize{resolution:module-resolution}}\index{resolution (module)}\index{get\_resolution() (in module resolution)}

\begin{fulllineitems}
\phantomsection\label{\detokenize{resolution:resolution.get_resolution}}\pysiglinewithargsret{\sphinxcode{\sphinxupquote{resolution.}}\sphinxbfcode{\sphinxupquote{get\_resolution}}}{\emph{img}}{}
Get the resolution of the image

Order (x,y) is swapped/reversed because the image format of TSI jpg files is reversed (for some reason). Resolution
in both directions is then set as global so that it can be called like:

\fvset{hllines={, ,}}%
\begin{sphinxVerbatim}[commandchars=\\\{\}]
\PYG{n+nb}{print}\PYG{p}{(}\PYG{n}{resolution}\PYG{o}{.}\PYG{n}{x}\PYG{p}{)}
\PYG{n+nb}{print}\PYG{p}{(}\PYG{n}{resolution}\PYG{o}{.}\PYG{n}{y}\PYG{p}{)}
\end{sphinxVerbatim}
\begin{quote}\begin{description}
\item[{Parameters}] \leavevmode
\sphinxstyleliteralstrong{\sphinxupquote{img}} (\sphinxstyleliteralemphasis{\sphinxupquote{int}}) \textendash{} Original unprocessed image

\end{description}\end{quote}

Returns:

\end{fulllineitems}



\section{settings module}
\label{\detokenize{settings:settings-module}}\label{\detokenize{settings::doc}}\label{\detokenize{settings:module-settings}}\index{settings (module)}
Set the global variables for:
\begin{itemize}
\item {} 
Directories

\item {} 
Aerosol corrections

\item {} 
Colors

\item {} 
Masking

\item {} 
Image segments

\item {} 
Sun features

\item {} 
GLCM calculations

\item {} 
Thresholding

\item {} 
Plotting

\item {} 
Toggling functions

\end{itemize}


\section{skycover module}
\label{\detokenize{skycover:skycover-module}}\label{\detokenize{skycover::doc}}\label{\detokenize{skycover:module-skycover}}\index{skycover (module)}\index{fixed() (in module skycover)}

\begin{fulllineitems}
\phantomsection\label{\detokenize{skycover:skycover.fixed}}\pysiglinewithargsret{\sphinxcode{\sphinxupquote{skycover.}}\sphinxbfcode{\sphinxupquote{fixed}}}{\emph{red\_blue\_ratio}, \emph{fixed\_sunny\_threshold}, \emph{fixed\_thin\_threshold}}{}
Calculate the fractional sky cover based on fixed thresholding.
\begin{quote}\begin{description}
\item[{Parameters}] \leavevmode\begin{itemize}
\item {} 
\sphinxstyleliteralstrong{\sphinxupquote{red\_blue\_ratio}} (\sphinxstyleliteralemphasis{\sphinxupquote{float}}) \textendash{} Pixel per pixel representation of the red/blue ratio

\item {} 
\sphinxstyleliteralstrong{\sphinxupquote{fixed\_sunny\_threshold}} (\sphinxstyleliteralemphasis{\sphinxupquote{float}}) \textendash{} clear sky/cloudy fixed threshold

\item {} 
\sphinxstyleliteralstrong{\sphinxupquote{fixed\_thin\_threshold}} (\sphinxstyleliteralemphasis{\sphinxupquote{float}}) \textendash{} thin/opaque fixed threshold

\end{itemize}

\item[{Returns}] \leavevmode
thin sky cover, opaque sky cover and fractional sky cover

\item[{Return type}] \leavevmode
tuple

\end{description}\end{quote}

\end{fulllineitems}

\index{hybrid() (in module skycover)}

\begin{fulllineitems}
\phantomsection\label{\detokenize{skycover:skycover.hybrid}}\pysiglinewithargsret{\sphinxcode{\sphinxupquote{skycover.}}\sphinxbfcode{\sphinxupquote{hybrid}}}{\emph{ratioBR\_norm\_1d\_nz}, \emph{hybrid\_threshold}}{}
Calculate the fractional sky cover based on hybrid thresholding.
\begin{quote}\begin{description}
\item[{Parameters}] \leavevmode\begin{itemize}
\item {} 
\sphinxstyleliteralstrong{\sphinxupquote{ratioBR\_norm\_1d\_nz}} (\sphinxstyleliteralemphasis{\sphinxupquote{float}}) \textendash{} normalized, masked, flattened red/blue ratio

\item {} 
\sphinxstyleliteralstrong{\sphinxupquote{hybrid\_threshold}} (\sphinxstyleliteralemphasis{\sphinxupquote{float}}) \textendash{} clear sky/cloud threshold determined by the hybrid algorithm

\end{itemize}

\item[{Returns}] \leavevmode
fractional sky cover as determined by the hybrid thresholding algorithm

\item[{Return type}] \leavevmode
float

\end{description}\end{quote}

\end{fulllineitems}



\section{statistical\_analysis module}
\label{\detokenize{statistical_analysis:statistical-analysis-module}}\label{\detokenize{statistical_analysis::doc}}\label{\detokenize{statistical_analysis:module-statistical_analysis}}\index{statistical\_analysis (module)}\index{work() (in module statistical\_analysis)}

\begin{fulllineitems}
\phantomsection\label{\detokenize{statistical_analysis:statistical_analysis.work}}\pysiglinewithargsret{\sphinxcode{\sphinxupquote{statistical\_analysis.}}\sphinxbfcode{\sphinxupquote{work}}}{\emph{maskedImg}}{}
Calculate the Grey Level Co-occurence Matrix (GLCM) and determine statistical features from it

This strategy is proposed by Heinle et al 2010

The statistical features can be used in machine learning algorithms such as k-nearest neighbor
\begin{quote}\begin{description}
\item[{Parameters}] \leavevmode
\sphinxstyleliteralstrong{\sphinxupquote{maskedImg}} (\sphinxstyleliteralemphasis{\sphinxupquote{int}}) \textendash{} masked RGB image

\item[{Returns}] \leavevmode
energy, entropy, contrast, homogeneity

\item[{Return type}] \leavevmode
tuple

\end{description}\end{quote}

\end{fulllineitems}



\section{thresholds module}
\label{\detokenize{thresholds:thresholds-module}}\label{\detokenize{thresholds:module-thresholds}}\label{\detokenize{thresholds::doc}}\index{thresholds (module)}\index{fixed() (in module thresholds)}

\begin{fulllineitems}
\phantomsection\label{\detokenize{thresholds:thresholds.fixed}}\pysiglinewithargsret{\sphinxcode{\sphinxupquote{thresholds.}}\sphinxbfcode{\sphinxupquote{fixed}}}{}{}
Get the fixed thresholds from the settings file
\begin{quote}\begin{description}
\item[{Returns}] \leavevmode
the fixed thresholds

\item[{Return type}] \leavevmode
tuple

\end{description}\end{quote}

\end{fulllineitems}

\index{flatten\_clean\_array() (in module thresholds)}

\begin{fulllineitems}
\phantomsection\label{\detokenize{thresholds:thresholds.flatten_clean_array}}\pysiglinewithargsret{\sphinxcode{\sphinxupquote{thresholds.}}\sphinxbfcode{\sphinxupquote{flatten\_clean\_array}}}{\emph{img}}{}
Convert 2D masked image to 1D flattened array to be used in MCE algorithm
\begin{quote}\begin{description}
\item[{Parameters}] \leavevmode
\sphinxstyleliteralstrong{\sphinxupquote{img}} (\sphinxstyleliteralemphasis{\sphinxupquote{int}}) \textendash{} masked image

\item[{Returns}] \leavevmode
normalized, 1D, flattened masked red/blue ratio array

\item[{Return type}] \leavevmode
float

\end{description}\end{quote}

\end{fulllineitems}

\index{hybrid() (in module thresholds)}

\begin{fulllineitems}
\phantomsection\label{\detokenize{thresholds:thresholds.hybrid}}\pysiglinewithargsret{\sphinxcode{\sphinxupquote{thresholds.}}\sphinxbfcode{\sphinxupquote{hybrid}}}{\emph{img}}{}
Decide between fixed or MCE thresholding as part of hybrid thresholding algorithm
\begin{quote}\begin{description}
\item[{Parameters}] \leavevmode
\sphinxstyleliteralstrong{\sphinxupquote{img}} (\sphinxstyleliteralemphasis{\sphinxupquote{int}}) \textendash{} masked image

\item[{Returns}] \leavevmode
normalized 1D flattened masked red/blue ratio array, standard deviation of the image and hybrid threshold

\item[{Return type}] \leavevmode
tuple

\end{description}\end{quote}

\end{fulllineitems}

\index{min\_cross\_entropy() (in module thresholds)}

\begin{fulllineitems}
\phantomsection\label{\detokenize{thresholds:thresholds.min_cross_entropy}}\pysiglinewithargsret{\sphinxcode{\sphinxupquote{thresholds.}}\sphinxbfcode{\sphinxupquote{min\_cross\_entropy}}}{\emph{data}, \emph{nbins}}{}
Minimum cross entropy algorithm to determine the minimum of a histogram
\begin{quote}\begin{description}
\item[{Parameters}] \leavevmode\begin{itemize}
\item {} 
\sphinxstyleliteralstrong{\sphinxupquote{data}} (\sphinxstyleliteralemphasis{\sphinxupquote{foat}}) \textendash{} the image data (e.g. blue/red ratio) to be used in the histogram

\item {} 
\sphinxstyleliteralstrong{\sphinxupquote{nbins}} (\sphinxstyleliteralemphasis{\sphinxupquote{int}}) \textendash{} number of histogram bins

\end{itemize}

\item[{Returns}] \leavevmode
the MCE threshold

\item[{Return type}] \leavevmode
float

\end{description}\end{quote}

\end{fulllineitems}



\chapter{Indices and tables}
\label{\detokenize{index:indices-and-tables}}\begin{itemize}
\item {} 
\DUrole{xref,std,std-ref}{genindex}

\item {} 
\DUrole{xref,std,std-ref}{modindex}

\item {} 
\DUrole{xref,std,std-ref}{search}

\end{itemize}


\renewcommand{\indexname}{Python Module Index}
\begin{sphinxtheindex}
\def\bigletter#1{{\Large\sffamily#1}\nopagebreak\vspace{1mm}}
\bigletter{c}
\item {\sphinxstyleindexentry{color\_bands}}\sphinxstyleindexpageref{color_bands:\detokenize{module-color_bands}}
\item {\sphinxstyleindexentry{createmask}}\sphinxstyleindexpageref{createmask:\detokenize{module-createmask}}
\item {\sphinxstyleindexentry{createregions}}\sphinxstyleindexpageref{createregions:\detokenize{module-createregions}}
\indexspace
\bigletter{l}
\item {\sphinxstyleindexentry{labelled\_image}}\sphinxstyleindexpageref{labelled_image:\detokenize{module-labelled_image}}
\indexspace
\bigletter{m}
\item {\sphinxstyleindexentry{main}}\sphinxstyleindexpageref{main:\detokenize{module-main}}
\item {\sphinxstyleindexentry{myimports}}\sphinxstyleindexpageref{myimports:\detokenize{module-myimports}}
\indexspace
\bigletter{o}
\item {\sphinxstyleindexentry{overlay}}\sphinxstyleindexpageref{overlay:\detokenize{module-overlay}}
\indexspace
\bigletter{p}
\item {\sphinxstyleindexentry{postprocessor}}\sphinxstyleindexpageref{postprocessor:\detokenize{module-postprocessor}}
\item {\sphinxstyleindexentry{processor}}\sphinxstyleindexpageref{processor:\detokenize{module-processor}}
\indexspace
\bigletter{r}
\item {\sphinxstyleindexentry{ratio}}\sphinxstyleindexpageref{ratio:\detokenize{module-ratio}}
\item {\sphinxstyleindexentry{read\_properties\_file}}\sphinxstyleindexpageref{read_properties_file:\detokenize{module-read_properties_file}}
\item {\sphinxstyleindexentry{resolution}}\sphinxstyleindexpageref{resolution:\detokenize{module-resolution}}
\indexspace
\bigletter{s}
\item {\sphinxstyleindexentry{settings}}\sphinxstyleindexpageref{settings:\detokenize{module-settings}}
\item {\sphinxstyleindexentry{skycover}}\sphinxstyleindexpageref{skycover:\detokenize{module-skycover}}
\item {\sphinxstyleindexentry{statistical\_analysis}}\sphinxstyleindexpageref{statistical_analysis:\detokenize{module-statistical_analysis}}
\indexspace
\bigletter{t}
\item {\sphinxstyleindexentry{thresholds}}\sphinxstyleindexpageref{thresholds:\detokenize{module-thresholds}}
\end{sphinxtheindex}

\renewcommand{\indexname}{Index}
\printindex
\end{document}